\chapter*{前言}

\par\noindent
% \marginnote{\color{cyan}%
% \zhdigits{12345}\zhdigits{67890}
% \zhdigits{12345}\zhdigits{67890}
% \zhdigits{12345}\zhdigits{67890}
% \zhdigits{12345}\zhdigits{67890}
% \zhdigits{12345}\zhdigits{67890}
% % 頭注字號{9.13}\rensuji{pt}\\\rensuji{@}11.869\rensuji{pt},
% % 行十字。 
% }
日本字,\CID{13674}\CID{1411}\UTF{7070}色「毛沢東」。“台湾”,電脳?\par
简体\UTF{4F53}字,\UTF{7070}色「毛\UTF{6CFD}\UTF{4E1C}」。“\UTF{53F0}\UTF{6E7E}”,\UTF{7535}\UTF{8111}?\par% 简体%
繁\UTF{7E41}體\UTF{9AD4}字,\UTF{7070}色「毛\UTF{6FA4}東」。“\UTF{81FA}\UTF{7063}”,電腦?\par% 繁体%
%ハングル,\UTF{7070}色 \UTF{d55c}\UTF{ae00} \par% 韩文%
% Table generated by Excel2LaTeX from sheet 'Sheet1'
森\UTF{9DD7}外と内田百\UTF{9592}とが\UTF{9AD9}島屋に行く。

\CID{7652}飾区の\CID{13706}野家,\CID{1481}城市,葛西駅,高崎と\CID{8705}\UTF{FA11}
  


\zhdigits{12345}\zhdigits{67890}
\zhdigits{12345}\zhdigits{67890}
\zhdigits{12345}\zhdigits{67890}
\zhdigits{12345}\zhdigits{67890}
\zhdigits{12345}\zhdigits{67890}
\zhdigits{12345}\zhdigits{67890}
\zhdigits{12345}\zhdigits{67890}
\zhdigits{12345}\zhdigits{67890}
\zhdigits{12345}\zhdigits{67890}
\zhdigits{12345}\zhdigits{67890}
\zhdigits{12345}\zhdigits{67890}
\zhdigits{12345}\zhdigits{67890}
\zhdigits{12345}\zhdigits{67890}
\zhdigits{12345}\zhdigits{67890}
\zhdigits{12345}\zhdigits{67890}
\zhdigits{12345}\zhdigits{67890}
\zhdigits{12345}\zhdigits{67890}
\zhdigits{12345}\zhdigits{67890}
\zhdigits{12345}\zhdigits{67890}
\zhdigits{12345}\zhdigits{67890}
\zhdigits{12345}\zhdigits{67890}
\zhdigits{12345}\zhdigits{67890}
\zhdigits{12345}\zhdigits{67890}
\zhdigits{12345}\zhdigits{67890}
\zhdigits{12345}\zhdigits{67890}
% \hspace{1zw}使用\verb+\LARGE+命令調用\dash\\
% 正文字號{15.521}\rensuji{pt}\rensuji{@}{27.39}\rensuji{pt},
% 一行\rensuji{31}字。正文與割注換算關係為:

\begin{center}
\noindent
一個正文字 = 1.67 割注字\footnote{脚注測試:序號為一}
\\%
\setcounter{footnote}{98}
一個正文字 = 2 行間注字\footnote{脚注測試:序號最大為九十九}
\endnote{需使用以下命令為行間注設置字號,使之恰好為正文字號的一半。}
% {\ttfamily $\backslash$rubyfontsetup\{$\backslash$mgfamily
% $\backslash$fontsize\{8.5pt\}\{10\}$\backslash$selectfont\} }\\[2mm]
% 另一個方法,自動模式:不設置字號,只設置字體風格,默認振假名為正文字號的一半。如:\\ \hspace{2zw}
% {\ttfamily $\backslash$rubyfontsetup\{$\backslash$mgfamily$\backslash$selectfont\} }}
\end{center}

\gezhu{
\zhdigits{12345}\zhdigits{67890}
\zhdigits{12345}\zhdigits{67890}
\zhdigits{12345}\zhdigits{67890}
\zhdigits{12345}\zhdigits{67890}\\
\zhdigits{12345}\zhdigits{67890}
\zhdigits{12345}\zhdigits{67890}
\zhdigits{12345}\zhdigits{67890}
\zhdigits{12345}\zhdigits{67890}\\
雙行割注字號{9.13}pt\faAt{9.13}pt in real diemen
}

% \clearpage
% \par\noindent{\normalsize
% 正文 normalsize,字號{9.13}\rensuji{pt}\rensuji{@}{16.434}\rensuji{pt},
% 行\rensuji{55}字。\\
% \zhdigits{12345}\zhdigits{67890}
% \zhdigits{12345}\zhdigits{67890}
% \zhdigits{12345}\zhdigits{67890}
% \zhdigits{12345}\zhdigits{67890}
% \zhdigits{12345}\zhdigits{67890}
% \zhdigits{12345}\zhdigits{67890}}


% \par\noindent{\fontsize{11pt}{12}\selectfont
% 測試字號{10.043}\rensuji{pt}\rensuji{@}{10.043}\rensuji{pt},
% 行\rensuji{50}字。\\
% \zhdigits{12345}\zhdigits{67890}
% \zhdigits{12345}\zhdigits{67890}
% \zhdigits{12345}\zhdigits{67890}
% \zhdigits{12345}\zhdigits{67890}
% \zhdigits{12345}\zhdigits{67890}
% \zhdigits{12345}\zhdigits{67890}}

\theendnotes
\newpage
