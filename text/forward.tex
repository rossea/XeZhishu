\chapter*{前言}
《长短经》是唐代西蜀赵蕤所著。据赵蕤自序,原为十卷六十三篇。后来第十卷《阴谋》缺失。今存九卷,计《文上》、《文中》、《文下》、《霸纪上》、《霸纪中》、《霸纪下》、《权议》、《杂说》和《兵权》,共六十四篇。

现存《长短经》最早的版本为南宋绍兴(1131-1162)年间刊本,清乾隆间修《四库全书》,用的就是这一版本。清代以来,《长短经》的重要版本还有《读画斋丛书》本(《丛书集成初编》本即据此排印)等。
国内出版的版本主要有周斌《<长短经>校证与研究》(巴蜀书社2003年版)、刘国建注译本《长短经》(长春出版社2001年版),以及张兆凯等注译本《长短经》(岳麓书社1999年版)等。



\newpage
\chapter*{自序}
\begin{flushright}
    赵蕤 
\end{flushright}
匠成舆者,忧人不贵;作箭者,恐人不伤。彼岂有爱憎哉?实技业驱之然耳。是知当代之士、驰骛之曹,书读纵横,则思诸侯之变;艺长奇正,则念风尘之会。此亦向时之论,必然之理矣。故先师孔子深探其本、忧其末,遂作《春秋》,大乎王道;制《孝经》,美乎德行。防萌杜渐,预有所抑。斯圣人制作之本意也。

然作法于理,其弊必乱。若至于乱,将焉救之?是以御世理人,罕闻沿袭。三代不同礼,五霸不同法。非其相反,盖以救弊也。是故国容一致,而忠文之道必殊;圣哲同风,而皇王之名或异。岂非随时投教沿乎此,因物成务牵乎彼?沿乎此者,醇薄继于所遭;牵乎彼者,王霸存于所遇。故古之理者,其政有三:王者之政化之;霸者之政威之;强国之政胁之。各有所施,不可易也。管子曰:“圣人能辅时不能违时。智者善谋,不如当时。”邹子曰:“政教文质,所以匡救也。当时则用之,过则舍之。”由此观之,当霸者之朝而行王者之化,则悖矣。当强国之世而行霸者之威,则乖矣。若时逢狙诈,正道陵夷,欲宪章先王,广陈德化,是犹待越客以拯溺,白大人以救火。善则善矣,岂所谓通于时变欤?

夫霸者,驳道也。盖白黑杂合,不纯用德焉。期于有成,不问所以;论于大体,不守小节。虽称仁引义不及三王,扶颠定倾,其归一揆。恐儒者溺于所闻,不知王霸殊略,故叙以长短术,以经论通变者,并立题目总六十有三篇,合为十卷,名曰《反经》。大旨在乎宁固根蒂,革易时弊,兴亡治乱。具载诸篇,为沿袭之远图,作经济之至道,非欲矫世夸欲,希声慕名。辄露见闻,逗机来哲。凡厥有位,幸望详焉。

